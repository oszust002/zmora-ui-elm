\chapter{Wprowadzenie} \label{chap:wprowadzenie}

W ciągu ostatnich kilku lat sposób tworzenia stron internetowych przechodził intensywne i szybkie zmiany. W związku z rosnącą popularnością internetu na całym świecie okazało się, że zwykłe, proste strony internetowe nie są wystarczające. W związku z tym szybko one awansowały ze statycznych dokumentów hipertekstowych jedynie wyświetlających zawartość użytkownikowi, do poziomu aplikacji internetowych, z którymi może on wejść w interakcję, a nawet używać ich dokładnie tak samo, jak programów zainstalowanych w swoim systemie operacyjnym. 

Rynek tworzenia oprogramowania webowego został przejęty głównie przez język JavaScript nazywany dziś przez niektórych asemblerem stron internetowych \cite{JSAssembly}. Coraz większe wymagania co do działania stron internetowych spowodowały, że tworzenie stron bezpośrednio w JavaScript'cie stało się zbyt skomplikowane i niewystarczające. Z pomocą przychodzą biblioteki, frameworki oraz języki kompilowane do JavaScriptu. Upraszczają one tworzony kod, często zmieniając też jego strukturę. Dzięki temu umożliwiają korzystanie z gotowych funkcjonalności w prosty i wygodny sposób. Przykładami tutaj mogą być biblioteki takie jak jQuery, Angular, React, Redux, czy Ember.js. 

W przypadku języków kompilowanych do JavaScriptu zmiany są jeszcze większe. Programista nie zastanawia się nad tym, że ostateczna wersja stworzonego przez niego programu jest zapisana w zupełnie innym języku. Przykładami takich języków są chociażby CoffeeScript, TypeScript czy Elm.

Ilość takich rozwiązań tworzy nieograniczoną liczbę podejść do tworzenia aplikacji webowych, a ich liczba każdego dnia rośnie. W związku z tym powstaje wiele artykułów porównujących różne podejścia, zarówno pod względami architektury kodu, jak i szybkości działania samych aplikacji.

Celem niniejszej pracy jest porównanie budowania i rozwoju aplikacji webowych przy pomocy języka Elm, oraz kombinacji bibliotek React.js i Redux. Technologie te zostaną szczegółowo porównane pod względem dostępnych funkcjonalności, szybkości działania aplikacji, dostępności bibliotek oraz trudności zarówno w tworzeniu pierwszej wersji aplikacji, jak i rozwoju istniejącego już kodu.

Wybór tematu pracy był podyktowany przede wszystkim poszukiwaniem alternatywnych rozwiązań służących do tworzenia aplikacji webowych. Biblioteki takie jak React czy Redux zdominowały rynek, przez co ilość wykorzystywanych sposobów budowania stron internetowych w stosunku do ilości dostępnych możliwości jest bardzo mała.

Ważną częścią pracy jest próba implementacji w języku Elm części projektu stworzonego przy pomocy kombinacji Reacta i Reduxa. Pozwoli to na praktyczną możliwość analizy obu rozwiązań i wyspecyfikowanie napotkanych problemów.

W rozdziale \ref{chap:teoria} zostały krótko opisane charakterystyka języka JavaScript oraz podstawowe pojęcia związane z tworzeniem aplikacji webowych. Następnie krótko zostały opisane biblioteki React i Redux oraz język Elm.
Rozdział \ref{chap:porownanie} skupia się na porównaniu architektury obu rozwiązań, zwracając uwagę zarówno na funkcjonalności, które są dostępne domyślnie zarówno w Elmie jak i kombinacji Reacta i Reduxa, jak i na te rozwiązania, które domyślnie oferuje tylko jedna ze stron.
W rozdziale \ref{chap:tools} opisane zostały narzędzia, które są oferowane przez oba porównywane rozwiązania, ich możliwości, ograniczenia, a także poziom trudności ich użycia.
Rozdział \ref{chap:implementacja} opisuje założenia implementacji projektu tworzonego w ramach pracy, oraz problemy napotkane podczas próby stworzenia aplikacji, domyślnie napisanej za pomocą Reacta i Reduxa, przy pomocy języka Elm.
Rozdział \ref{chap:podsumowanie} zawiera krótkie podsumowanie przeprowadzonej analizy porównawczej. Opisuje on wnioski oraz możliwości rozwoju pracy.
