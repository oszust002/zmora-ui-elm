\chapter{Porównanie architektury} \label{chap:porownanie}
Choć porównywanie kombinacji bibliotek z językiem programowania wydaje się dziwne, to jeżeli spojrzy się na funkcjonalności oferowane przez Reacta i Reduxa w porównaniu do możliwości Elma, można zauważyć pewne podobieństwa w sposobie budowania aplikacji. W tym rozdziale zostały opisane funkcjonalności, które są oferowane przez oba rozwiązania, oraz różnice występujące między Reactem i~Reduxem a Elmem dla każdej z funkcjonalności.

\section{Virtual DOM i składnia}
\subsection{Document Object Model}
DOM \footnote{z. ang. Document Object Model} jest niezależnym od platformy i języka programowania interfejsem, który pozwala programom i skryptom na dynamiczny dostęp i aktualizację treści, struktury i stylu dokumentu. Kiedy strona internetowa jest ładowana, przeglądarka tworzy DOM strony, będący obiektową reprezentacją dokumentu HTML. Służy ona jako interfejs umożliwiający pobieranie oraz modyfikację elementów HTML, które w DOM-ie są zdefiniowane jako obiekty.

\subsection{Dlaczego DOM jest powolny?}
Każda akcja na stronie powoduje zmianę DOM-u. Ze względu na jego drzewiastą strukturę, sama modyfikacja DOM-u jest szybka. Jednak każdy z modyfikowanych elementów oraz wszystkie jego dzieci muszą dodatkowo przejść przez dwa kosztowne etapy:
\begin{enumerate}
	\item Reflow będący procesem, podczas którego przeliczane zostają wymiary oraz pozycja elementu. Dokładnie ten sam proces jest uruchamiany na węzłach dzieci, a także elementach, które pojawiają się w DOM-ie później niż główny element. Reflow jest kosztowny, ponieważ zmiana pojedynczego elementu w strukturze DOM-u może spowodować wywołanie Reflow na wielu innych elementach.
	\item Repaint, w którym niektóre partie ekranu muszą zostać zaktualizowane, czy to ze względu na modyfikacje wymiarów i~pozycji elementu, czy przez zmiany stylistyczne, takie jak zmiana koloru tła. Etap ten jest kosztowny ponieważ przeglądarka musi sprawdzić widoczność innych węzłów w~DOM-ie.
\end{enumerate} 

\subsection{Virtual DOM}
Virtual DOM jest to lekka, niezależna od przeglądarki abstrakcyjna reprezentacja DOM-u. Służy ona między innymi do zminimalizowania kosztu stworzonego przez etapy Reflow i Repaint. Zamiast tworzyć za każdym razem drzewo składające się z węzłów DOM-u, Virtual DOM pozwala na stworzenie tego drzewa przy pomocy abstrakcyjnych węzłów, które są odpowiednikami faktycznych elementów DOM-u. Dzięki temu wszystkie operacje modyfikujące widok mogą być wykonane na abstrakcyjnej strukturze, a dopiero końcowy rezultat powoduje modyfikację faktycznego DOM-u.

\subsection{Algorytm porównywania różnic}
Koncepcja Virtual DOM-u została wykorzystana do tego, aby w każdej klatce budować zupełnie nową scenę. Choć taka operacja wydaje się kosztowna, to tak naprawdę zbudowanie pełnego drzewa Virtual DOM-u jest szybkie, i jest wykorzystywane przy każdej aktualizacji widoku. W momencie gdy następuje zmiana powodująca modyfikację widoku, algorytm porównuje ze sobą stare i nowe drzewo Virtual DOM-u. Wszystkie komponenty, w których nastąpiła jakakolwiek zmiana są oznaczane specjalną flagą, która określa, że dany węzeł został zmodyfikowany. Na tej podstawie budowana jest dokładna lista zmian jakie nastąpiły w widoku. Następnie lista ta jest wykorzystywana do modyfikacji faktycznego DOM-u, lecz nie jako pojedynczo wprowadzane zmiany, a jedna aktualizacja drzewa dokumentu. 

\begin{figure}[h]
	\centering
	\includegraphics[width=0.8\textwidth]{images/diff_algorithm}
	\caption{Schemat działania algorytmu  porównywania różnic}
	\label{fig:diffAlgorithm}
\end{figure}

\subsection{Reprezentacja w Elmie}
Zarówno React, jak i Elm posiadają własne implementacje Virtual DOM-u. W Elmie abstrakcyjną reprezentację węzła otrzymujemy przy pomocy funkcji \lstinline{node}, która jako atrybuty przyjmuje tag, listą atrybutów HTML, oraz listę węzłów dzieci:
\begin{lstlisting}[style=elm-style]
	node : String -> List Attrbiute -> List Html -> Html
\end{lstlisting}
W przypadku użycia tagów HTML, takich jak \lstinline{div}, Elm udostępnia funkcje pomocnicze, które posiadają już uzupełniony atrybut tag, pozostawiając nam do określenia atrybuty węzła oraz jego dzieci. Elm nie ma specjalnej składni, która służyłaby do budowania widoków. Wszystkie elementy, z których budowany jest widok aplikacji, są funkcjami. W przypadku budowania widoku jedynym wymaganiem jest, aby funkcja go budująca zwracała rekord specjalnego typu \lstinline[style=elm-style]{Html msg}, który jest głównym blokiem służącym do tworzenia wyjściowego kodu HTML.

\subsection{Reprezentacja w React}
W przypadku Reacta mamy do czynienia ze znacznie bardziej rozszerzonym podejściem.  Bazowym elementem reprezentującym abstrakcyjny węzeł Virtual DOM-u jest ReactElement. Analogicznie jak w przypadku funkcji \lstinline{node} w Elmie jest to obiekt posiadający informację o tagu, który reprezentuje, atrybutach zdefiniowanego węzła, oraz listę dzieci. Przykład tworzenia takiego elementu można zobaczyć we fragmencie kodu \ref{listing:jsreactelement}. 

\begin{minipage}{.45\textwidth}
	\begin{lstlisting}[caption=Javascript,style=JavaScript,label = listing:jsreactelement]
	var divHello = React.createElement(
	"div",
	{ className: "myclass" },
	"Hello world!"
	);
	\end{lstlisting}
\end{minipage}\hfill
\begin{minipage}{.45\textwidth}
	\begin{lstlisting}[caption=JSX,style=JavaScript,firstnumber=1,label = listing:jsx]
	var divHello = (
	<div className="myclass">
		Hello world!
	</div>
	);
	\end{lstlisting}
\end{minipage}

\subsection{Test wydajnościowy}
Biorąc pod uwagę to, że każdy element Virtual DOM-u w React'cie jest zbudowany w podobny sposób, można zauważyć, że w przypadku rozbudowanej aplikacji, kod bardzo szybko staje się skomplikowany i niezrozumiały. Aby tego uniknąć, Facebook stworzył specjalne rozszerzenie składni dla JavaScriptu -- JSX. Z wyglądu przypomina składnię języka HTML, lecz zasadniczo dostarcza cukier syntaktyczny dla funkcji \lstinline[style=JavaScript]{createElement}. Fragmenty kodu \ref{listing:jsx} oraz \ref{listing:jsreactelement} są dokładnie tymi samymi wyrażeniami, z tą różnicą, że w drugim przypadku został wykorzystany JSX. Takie podejście pozwala na użycie JSX-a wewnątrz instrukcji JavaScriptu, przypisywanie go do zmiennych, czy też zwracanie z funkcji. Operacja zachodzi także w drugą stronę, co znaczy, że można korzystać z kodu JavaScriptu wewnątrz składni JSX-a. Taki kod musi być objęty nawiasami klamrowymi, aby odróżnić fragmenty napisane w JavaScript'cie od kodu JSX-a. 

% Test czasu renderowania
Choć główne założenia Virtual DOM-u w Elmie i React'cie są podobne, to jego implementacje nie są identyczne, w związku z czym można je porównać pod względem wydajnościowym. W tym przypadku wykorzystany został test wydajnościowy stworzony przez twórcę Elma \cite{perComp}. Pozwala on na porównanie czasu renderowania różnych implementacji aplikacji TodoMVC. Jest to prosty projekt listy zadań, umożliwiający dodawanie i usuwanie wpisów, oznaczanie ich jako zakończone, a także odfiltrowywanie na podstawie statusu wpisów. Test zakłada realistyczny scenariusz, w którym każda zmiana jest wyświetlona jako pojedyncza klatka, tak jakby to faktyczny użytkownik przeprowadzał test. Algorytm scenariusza wykonywany w trakcie pomiaru średniego czasu renderowania wygląda następująco:
\begin{enumerate}
	\item Stworzenie strony niezawierającej wpisów
	\item Dodanie 100 wpisów do listy
	\item Oznaczenie każdego z elementów jako zakończony
	\item Usunięcie wszystkich wpisów
\end{enumerate}
Dodatkowo zostały przyjęte następujące założenia, które sprawiają, że przeprowadzony test jest sprawiedliwy:
\begin{enumerate}
	\item Brak zgrupowanych zdarzeń -- oznacza to, że zamiast generować zdarzenia w pojedynczej pętli, algorytm tworzy jedno zdarzenie na raz, przechodząc do następnego dopiero po wyrenderowaniu całego widoku. Gdyby takie założenie nie zostało przyjęte, to przykładowo w przypadku dodawania wpisów, zmiany następowałyby na tyle szybko, że zamiast wyświetlić 100 klatek, przeglądarka wyświetliłaby tylko jedną.
	\item Brak użycia \lstinline{requestAnimationFrame} -- funkcja ta informuje przeglądarkę o zamiarze wykonania animacji i żąda od przeglądarki wywołania określonej funkcji w celu odświeżenia animacji przed następną zmianą w widoku. Oznacza to, że odświeżenie animacji jest wyrównane do 60 razy na sekundę, niezależnie od tego, jak wiele klatek wygeneruje JavaScript. Elm wykorzystuje tą funkcję do pomijania części klatek, które i tak nie będą widoczne dla użytkownika. W z związku z realistycznym scenariuszem, oraz brakiem podobnej optymalizacji w innych implementacjach, funkcja ta musiała zostać usunięta z Elma w ramach przeprowadzanego testu.
\end{enumerate}

\begin{figure}[h]
	\centering
	\includegraphics[width=0.9\textwidth]{images/render_comparision}
	\caption{Porównanie czasu renderowania aplikacji TodoMVC (w oparciu o \cite{perComp})}
	\label{fig:performanceComparision}
\end{figure}

Na rysunku \ref{fig:performanceComparision} można zauważyć, że zostały wzięte pod uwagę dwie wersje Elma. Powodem jest tutaj zmiana używanej implementacji Virtual DOM-u. Od początku istnienia Elma aż do wersji 0.16, wykorzystywana była implementacja Matta Escha, która była silnie inspirowana wersją Virtual DOM-u wykorzystywaną w React'cie. Jednak z powodu dużych zmian wprowadzonych w nowszych wersjach Elma, twórca języka był zmuszony stworzyć własną implementację dopasowaną do nowego API. 

Wyniki testu pokazują, że implementacja aplikacji w Elmie jest szybsza o ponad sekundę. Twórca Elma w jednym ze swoich artykułów \cite{blazingFastHtml} pisze o wykorzystanych technikach, które są powodem takich wyników:
\begin{enumerate}
	\item Używanie tablic zamiast słowników -- iteracja po tablicy jest zawsze o wiele szybsza operacją niż przechodzenie po kluczach słownika.
	\item Minimalizacja ilości alokacji -- garbage collection jest jednym z kosztownych elementów w analizowanych implementacjach. Im mniej obiektów jest alokowanych, tym lepsza jest wydajność aplikacji. Sposób wykorzystany w Elmie polega na alokowaniu obiektów z pustymi polami. Dzięki temu silniki JavaScriptowe radzą sobie o wiele lepiej z optymalizacją takich obiektów, a obiekt nie zmienia się, nawet gdy wypełnimy go większą ilością informacji.
	\item Unikanie powolnych operacji, takich jak pobieranie konkretnego elementu z tablicy. 
\end{enumerate}

\section{Jednokierunkowy przepływ danych}
\subsection{Architektura języka Elm}
Jednokierunkowy przepływ danych w języku Elm jest często określany jako \textit{Architektura języka Elm} lub \textit{Model-View-Update}. Niezależnie od rozmiaru tworzonej aplikacji, każdą z nich można podzielić na trzy całkowicie oddzielone od siebie części:
\begin{itemize}
	\item Model -- jest strukturą danych zawierającą wszystkie informacje o stanie aplikacji. W programie jest reprezentowany jako rekord o ściśle określonym typie.
	\item Update -- sposób, w jaki stan aplikacji jest aktualizowany, reprezentowany przez funkcję przyjmującą jako argumenty komunikat o rodzaju aktualizacji danych oraz dotychczasowy stan aplikacji, a zwracającą nowy, zaktualizowany stan.
	\item View -- definicja sposobu wyświetlania stanu aplikacji w formie kodu HTML. Jest reprezentowany przez funkcję przyjmującą jako argument aktualny stan aplikacji, a zwracającą reprezentację kodu HTML w formie obiektów Virtual DOM-u.
\end{itemize}
\begin{figure}[h]
	\centering
	\includegraphics[width=0.6\textwidth]{images/elm_data_flow}
	\caption{Przepływ danych w języku Elm}
	\label{fig:elmFlow}
\end{figure}
\FloatBarrier
Niezależnie od rodzaju wykonywanej akcji, przepływ danych w języku Elm opiera się o podstawowe struktury zwane wiadomościami. Są to komunikaty zdefiniowane przez programistę w kodzie, które poza typem wiadomości mogą zawierać dane, na podstawie których aktualizowany jest modek. Silnik języka przesyła je wraz z aktualną wersją modelu do funkcji \lstinline{update}, która na podstawie rodzaju wiadomości oraz zawartych w niej danych aktualizuje model. Nowa wersja modelu zostaje następnie przekazana do funkcji \lstinline{view}, która buduje na jego podstawie pełną strukturę opisującą widok. Struktura ta, będąca obiektem typu \lstinline[style=elm-style]{Html}, skonstruowanym przy pomocy funkcji z implementacji Virtual DOM-u, jest następnie przesyłana do silnika Elma, który na jej podstawie aktualizuje drzewo DOM-u.

Elm posiada także dwa dodatkowe rodzaje komunikatów: komendy i subksrypcje. Te pierwsze odpowiadają za wysyłanie wiadomości obsługujących efekty uboczne, takie jak zapytania HTTP. Są one tworzone jako obiekty typu \lstinline[style=elm-style]{Cmd} posiadające w sobie typ wiadomości, jaka ma zostać wywołana. Obiekty takie są wysyłane równocześnie z modelem, jako część zwrotna funkcji \lstinline{update}, która wysyła model do funkcji odpowiadającej za budowę widoku, natomiast samą komendę przekazuje do silnika Elma. Silnik na podstawie podanej komendy wywołuje kolejną aktualizację modelu, jako wiadomość przekazując zawartość komendy. Subskrypcje natomiast, są sposobem na nasłuchiwanie zewnętrznych komunikatów takich jak ruchy myszy, zmiana adresu w przeglądarce, czy też zmiana jej rozmiaru. Są one zdefiniowane podczas uruchomienia aplikacji jako obiekty typu \lstinline[style=elm-style]{Sub}, które podobnie jak w przypadku komend posiadają w sobie typ wiadomości, która po nastąpieniu zewnętrznego zdarzenia jest wysyłana przez silnik języka do funkcji \lstinline{update}.

Prostota tak skonstruowanego przepływu danych pozwala na łatwą analizę działania aplikacji, niezależnie od jej rozmiaru. Programista nie jest zmuszony do zastanawiania się nad skomplikowaną architekturą, dzięki czemu jest w stanie szybko ustalić, w jaki sposób działa aplikacja. Prosty model aktualizacji danych oparty na komunikatach pozwala także na łatwe dodawanie nowych funkcjonalności, ponieważ wiąże się to z dodaniem nowego komunikatu, który jest definiowany niezależnie od poprzednio dodanych wiadomości.

\subsection{Komponent prezentacyjny i kontener}
Aby w ogóle zacząć temat jednokierunkowego przepływu danych w kombinacji React i Redux, trzeba wyjaśnić, czym są komponenty. W przeciwieństwie do Elma, który całą swoją strukturę opiera na funkcjach, React wprowadza specjalny rodzaj struktur umożliwiający zdefiniowanie wyświetlanych widoków.

W dokumentacji Reacta można znaleźć, że komponenty pozwalają podzielić interfejs na niezależne fragmenty wielokrotnego użytku i myśleć o każdym z fragmentów oddzielnie \cite{reactDocs}. Koncepcyjnie są one podobne do funkcji w JavaScript'cie. Przyjmują dowolne dane wejściowe, które zawarte są w pojedynczym obiekcie \lstinline{props}, a zwracają elementy Reacta, opisujące to, co powinno zostać wyświetlone. Fragmenty \ref{listing:statelesscomp} i \ref{listing:statefulcomp} przedstawiają dwa podstawowe sposoby tworzenia komponentów w React'cie: jako zwykła funkcja JavaScript oraz jako klasa ze standardu ECMAScript 6.
\newpage % Simple hack to remove first line of listing from page end
\begin{lstlisting}[style=JavaScript, caption=Funkcyjny komponent bezstanowy, label=listing:statelesscomp]
function Welcome(props) {
	return <h1>Hello, {props.name}</h1>;
}
\end{lstlisting}

\begin{lstlisting}[style=JavaScript, caption=Komponent stanowy jako klasa ECMAScript 6, label=listing:statefulcomp]
class Clock extends React.Component {
	constructor(props) {
		super(props);
		this.state = {date: new Date()};
	}
	render() {
		return (
			<div>
				<h1>It is {this.state.date.toLocaleTimeString()}.</h1>
			</div>
		);
	}
}
\end{lstlisting}
W pierwszym przypadku mamy do czynienia z komponentem funkcyjnym bezstanowym. Zgodnie z~jego nazwą charakteryzuje się brakiem lokalnego stanu oraz zapisem w formie funkcji przyjmującej jako argument obiekt \lstinline{props}, która zwraca elementy Reacta. Komponent tego typu nie pozwala także na zarządzanie jego cyklem życia, czy optymalizacją częstotliwości odświeżania. Aby skorzystać z tych funkcjonalności, należy użyć implementacji za pomocą klasy. Komponenty klasowe umożliwiają dodanie logiki do cyklu życia komponentu, co sprawia, że stają się one czymś więcej niż tylko obiektami opisującymi jakie elementy mają zostać wyświetlone na interfejsie użytkownika.

W kontekście połączenia Reacta i Reduxa często stosuje się inny podział, na komponenty prezentacyjne i kontenery. Różnią się one od siebie przede wszystkim tym, że kontenery są specjalnym rodzajem komponentów, które są świadome istnienia Reduxa. To one odpowiadają między innymi za pobranie danych ze stanu Reduxa, czy wysyłaniu akcji określających, w jaki sposób ma zostać zmieniony stan. Pobieranie danych w kontenerze się to poprzez użycie funkcji \lstinline{connect} ze specjalnej biblioteki \textit{react-redux} zawierającej kod wymagany do współpracy Reacta z Reduxem \cite{reduxDocs}. Funkcja \lstinline{connect} przyjmuje jako argument zdefiniowaną przez użytkownika funkcję określającą, który fragment stanu Reduxa ma zostać użyty w kontenerze. W przypadku komponentów prezentacyjnych jedynym źródłem danych są właściwości przekazane za pomocą obiektu \lstinline{props}, a jedynym możliwym sposobem na zmianę stanu są funkcje zwrotne, również przekazane jako część tego obiektu.

Powodem takiego podziału jest przede wszystkim oddzielenie komponentów widoku, które mogą być używane wielokrotnie i nie powinny zależeć od implementacji logiki odpowiadającej za pobieranie i zarządzanie danymi.

\subsection{Przepływ danych w Redux i lokalny stan komponentów}
Twórca Reduxa tworząc implementację biblioteki, inspirował się między innymi Elmem, w związku z~czym sposób przepływu danych występujący w bibliotece Redux jest bardzo zbliżony do implementacji z języka Elm, a pewne fragmenty Reduxa można porównać do części architektury \textit{Model-Update-View}.

Odpowiednikiem modelu jest pojedyncze drzewo stanu zawierające w sobie wszystkie dane aplikacji. Różnica występująca między implementacją w Elmie jest brak ograniczenia co do typu, co pozwala na dowolną modyfikację drzewa, bez zbędnej potrzeby wcześniejszego definiowania domyślnych wartości. 

Tak jak w Elmie podstawową strukturą służącą do komunikowania o aktualizacji jest wiadomość, tak w przypadku Reduxa wykorzystywane są akcje. Są to zwykłe obiekty JavaScriptu posiadające informacje o rodzaju komunikatu, a także mogące posiadać dane wykorzystane do aktualizacji. W przeciwieństwie do Elma Redux nie posiada dodatkowych komunikatów służących do obsługi efektów ubocznych czy zdarzeń z zewnątrz, co zmusza programistę do skorzystania z innych bibliotek obsługujących takie zdarzenia. Przykładem tutaj może być biblioteka \textit{redux-saga}, służąca do obsługi efektów ubocznych \cite{reduxSaga}, która dodatkowo wspiera komunikację z Reduxem.

Aktualizacja danych odbywa się za pomocą funkcji zwanych \textit{reducer'ami}. Są to funkcje bez efektów ubocznych, które na podstawie dotychczasowego stanu oraz akcji zwracają nowe, zaktualizowane drzewo stanu. W przypadku Reduxa programista nie jest zmuszony do tworzenia jednej funkcji odpowiadającej za wszystkie aktualizacje. Biblioteka udostępnia funkcję \lstinline{combineReducers}, przyjmującą wszystkie funkcje, które mają posłużyć jako reducer. Po przysłaniu akcji funkcja przesyła ją do wszystkich określonych w niej funkcji aktualizujących, a następnie zbiera wyniki do jednego obiektu.
\begin{figure}[h]
	\centering
	\includegraphics[height=0.38\textheight]{images/react_redux_data_flow}
	\caption{Przepływ danych w kombinacji React i Redux}
	\label{fig:reactReduxFlow}
\end{figure}
\FloatBarrier


\section{Niemutowalność obiektów}

\section{Biblioteki i menedżery pakietów}
