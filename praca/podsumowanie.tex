\chapter{Podsumowanie} \label{chap:podsumowanie}
Celem tej pracy było porównanie dwóch sposobów budowania i rozwoju aplikacji internetowych: przy pomocy języka Elm oraz kombinacji bibliotek React.js i Redux. 
Obie technologie miały zostać porównane pod względem architektury, udostępnianych funkcjonalności, wydajności aplikacji oraz dostępnych bibliotek i narzędzi. Dodatkowo w ramach praktycznego porównania miała zostać przygotowana aplikacja stworzona przy pomocy języka Elm, która odwzorowywała pewną część innej aplikacji napisanej przy pomocy bibliotek React i Redux. 

Pomimo porównania języka do dwóch bibliotek, okazało się, że pod względem architektonicznym połączenie Reacta i Reduxa jest bardzo podobne do podejścia stosowanego w języku Elm. Te same funkcjonalności zaimplementowane w różny sposób pozwoliły na porównanie, która z technologii może być oceniana jako lepsza w kontekście tworzenia aplikacji internetowych. Sporym czynnikiem mającym wpływ na ocenę obu technologii były problemy, które pojawiły się podczas realizacji praktycznej części pracy w formie projektu. 

Analiza architektury języka Elm oraz bibliotek React i Redux pokazała, że w zależności od wybranego kontekstu, na podstawie którego te technologie są porównywane, decyzja o tym, która z nich jest lepsza, potrafi być zmienna. 

Pod względem szybkości wyświetlania widoku zdecydowanym zwycięzcą jest Elm, który dzięki optymalizacjom zawartym w swoim silniku do budowy Virtual DOM-u jest w stanie pominąć niektóre klatki, które i tak nie są widoczne dla użytkownika. Argumentem stojącym za wyborem Elma jest też to, że pewne elementy, takie jak narzędzia do analizy kodu i procesu działania aplikacji są wbudowane bezpośrednio w sam silnik języka, w przeciwieństwie do Reacta i Reduxa, które muszą korzystać z zewnętrznych rozwiązań, o których trzeba pamiętać. Część porównywanych elementów architektury nie pozwala jednak na bezpośrednie wybranie lepszej technologii, ponieważ każdy programista preferuje inne podejście do tworzenia oprogramowania. Mowa tu o elementach takich jak wybór między silnym a~dynamicznym typowaniem, preferencji trzymania wszystkich danych w jednym miejscu, a możliwością skorzystania z lokalnego stanu oferowanego przez komponenty, czy chociażby wybór między komponentami zawierającymi własny cykl życia a funkcjami w Elmie działającymi za każdym razem w ten sam sposób, bez efektów ubocznych. Aspektem, który stawia Reacta i Reduxa na wygranej pozycji jest przede wszystkim ich popularność oraz dostępność bibliotek. Z racji tego, że są to po prostu biblioteki języka JavaScript, mogą one korzystać bezpośrednio z każdego dostępnego pakietu stworzonego przy pomocy tego języka, który jest obecnie najpopularniejszym językiem do budowania aplikacji internetowych.

Podsumowując całą pracę, nie można wybrać, która z porównywanych technologii jest lepsza. Zarówno język Elm, jak i kombinacja Reacta i Reduxa mają swoje zalety, a także wady. Trzeba zwrócić także uwagę na to, że porównując technologie wykorzystywane do tworzenia aplikacji internetowych, wyników takiego porównania nie można brać pod uwagę przez dłuższy okres. Rynek tworzenia oprogramowania webowego jest jednym z najczęściej i najszybciej zmieniających się gałęzi programowania. Być może za kilka lat zmieni się on na tyle, że jego popularność wzrośnie i całkowicie wyprze używane obecnie biblioteki. 